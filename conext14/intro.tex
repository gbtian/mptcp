%!TEX root =mpip.tex
\section{Introduction}
\label{sec:intro}
Multipath has become available in recent years. Also, IETF proposed RFC $6182$ specifically for multipath TCP in $2011$. By introducing multipath, not only higher throughput can be achieved, the reliability of connections also benefits because multipath on one connection can achieve failover, i.e., if one path fails, the others will keep working without influencing applications.

Most current devices (Mainly mobile devices) have more than one internet interface ($4$G, WiFi), it is possible to make use of this facility to improve the quality of Internet transmission. In scenarios that end users want high throughput, parallel multipath transmission can possibly improve throughput. In scenarios that end users have intermittent internet connection on one interface, multipath connection can provide smooth switching between connections.

Current work on multipath is mainly on TCP. In MPTCP, if the user has more than one Internet interface, there will be more than one subflow in one TCP connection while each subflow is an independent TCP connection. In this way, the user does not need to re-establish the connection when switching connection. But MPTCP can only be used in TCP connection given that there is still large amount of non-TCP traffic on the Internet although TCP traffic is dominating.

To enable non-TCP traffic to use multipath feature, we propose our multipath implementation at network layer. Introducing multipath at network layer has several benefits over transportation layer. At transportation layer, the complexity of TCP protocol makes the design of multipath over TCP more complicated and vulnerable. These built-in characters of MPTCP predesignate that MPTCP is not flexible enough to satisfy various requirement of different applications. As for MPIP, the simplicity of IP protocol makes it much easier to implement. And almost all traffic on the Internet will go through IP protocol, this makes MPIP more omnipotent than MPTCP. As will show in later sections, because of heterogeneous characters of different NICs on one device, different configuration can generate totally different user experience. If MPIP can provide equivalent performance with MPTCP for TCP traffic, then the good flexibility of MPIP definitely provides a much better option for multipath deployment.

By implementing the prototype of MPIP, our contribution is three-fold.
\begin{enumerate}
\item We propose the overall design and architecture of MPIP. By comparing our design with MPTCP, we see that implementing multipath at network layer has much lighter weight and more straightforward than transportation layer.

\item We implement our design in the latest Linux kernel under Ubuntu system. Also, we evaluate the implementation in different Internet environments. We show that our implementation can match MPTCP in TCP protocol, and also, other protocols like UDP can fit perfectly with multipath IP.

\item For investigation purpose, we combine the implementation of MPTCP and MPIP together to prove multipath feature at both layers. It turns out that this combination can provide better and more consistent performance over certain Internet conditions.
\end{enumerate}

The rest of the paper is organized as follows. Section \ref{sec:related} describes the related work.
The design of our implementation is introduced in Section \ref{sec:design}. In Section \ref{sec:evaluation}, we report the experimental results for our multipath IP design. We conclude the paper with summary and future work in Section \ref{sec:conclusion}.
